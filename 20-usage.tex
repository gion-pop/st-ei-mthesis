\chapter{クラスファイルについて}
\label{chap:usage}

\section{BXjscls由来の設定・コマンド}
\subsection{クラスオプション}
BXjsclsパッケージをベースに用いているので,~\cite{bxjscls-manual}を参考に
必要なクラスオプションを取捨選択してほしい.

\paragraph{{p\LaTeX}の場合}
プリアンブルに
\verb|\documentclass[dvipdfmx,platex]{st-ei-mthesis}|
と記述.

\paragraph{{Lua\LaTeX}の場合}
プリアンブルに
\verb|\documentclass[lualatex]{st-ei-mthesis}|
と記述.


\subsection{コマンド}
BXjsclsパッケージが提供するもののうち,
ユーザーにとって重要そうなもののみについて簡単な説明をする.

\paragraph{setpagelayout}
余白などの用紙使いについて設定できる.
書式については\texttt{geometry}パッケージのものをそのまま使えるので,
パッケージのドキュメントやWebの記事などを参考に調整すればよい.


\section{新たに定義したコマンド}

\subsection{タイトルページ用のコマンド}
\label{sec:title-cmd}
\begin{itemize}
\item
  \texttt{jyear}: ○○年度の○○の部分.
  和暦とか西暦とか両方考えるのが面倒だったので使う側に任せる.
\item
  \texttt{jtitle}: 論文のタイトル.
\item
  \texttt{jauthor}: 著者名.
\item
  \texttt{jsupervisor}: 指導教員.
\item
  \texttt{jexaminer}: 主査.
\item
  \texttt{jdate}: 日付.
\end{itemize}
以上を設定した状態で\verb|\jmaketitle|によってタイトルページが生成できる.


\subsection{要旨ページ用のコマンド}
\ref{sec:title-cmd}節で説明したもののうち
\texttt{jyear}, \texttt{jtitle},\texttt{jauthor}が設定してある状態で
\verb|\begin{jabstract}|〜\verb|\end{jabstract}|環境内に
要旨の文章を記述すると要旨ページが作成される.
このときあらかじめ\texttt{jkeywords}にキーワード群が設定されていれば,
要旨の下にキーワード一覧が出力される.
見出しを『論文要旨』以外の文言に変更したければ
\verb|\begin{jabstract}[要旨]|などとすればよい.


\subsection{謝辞用環境}
\verb|\begin{acknowledgement}|〜\verb|\end{acknowledgement}|環境内に
謝辞を書ける.
要旨環境と同様にオプションで見出しを変更できる.


\section{試し刷りとか添削とか}
\verb|\documentclass[platex,dvipdfmx,oneside,openany]{st-ei-mthesis}|
などとすれば余分な空白ページが出力されない.
\verb|\setpagelayout{hmargin=20mm}|などとして
左右のマージンを同じにしてしまってもよいかも.